\documentclass[a4paper,11pt]{scrartcl}
\usepackage[ngerman]{babel}
\usepackage[utf8]{inputenc}
\usepackage[T1]{fontenc}
\usepackage{lmodern}
\usepackage{microtype}

%opening
\title{Zeitkomplexitäten der Teilaufgaben}
\author{Matrikelnummer 381165}

\begin{document}
    
    \maketitle
    
    \section{Erstellung und Ausgabe der Dominosteinliste}
    
    \begin{itemize}
        \item es wird stets die Datei einmal komplett durchlaufen und eingelesen: $O(n)$
        \begin{itemize}
            \item dabei wird mit konstanter Anzahl an Operationen bei jeder zweiten Zahl ein Listenelement erstellt: $O(1)$
        \end{itemize}
        \item anschließend wird die Liste stets komplett lesend durchlaufen, um die Elemente darzustellen: $O(n)$
        \item Zeitkomplexität der ganzen Teilaufgabe: $O(n \cdot 1 + n) \Rightarrow{} O(n)$
    \end{itemize}
    
    \section{Erstellung der Kreise}
    
    \begin{itemize}
        \item es werden insgesamt genau so viele Kreiselemente erstellt wie Listenelemente aus Aufgabe 1 vorhanden sind: $O(n)$
        \begin{itemize}
            \item dabei muss immer ein variabler Teil der verbleibenden Dominoliste durchiteriert werden: $O(\frac{n}{2}) \Rightarrow{} O(n)$
        \end{itemize}

        \item Zeitkomplexität der ganzen Teilaufgabe: $O(n \cdot \frac{n}{2}) \Rightarrow{} O(n^2)$
    \end{itemize}
    
    \section{Ausgabe der Kreise}
    
    \begin{itemize}
        \item  es werden ggf. mehrere Ringlisten nacheinander stets einmal komplett durchlaufen und die Elemente dabei ausgegeben: $O(n)$
        \item dabei ist die Summe der insgesamt ausgegebenen Kreiselemente gleich der der Listenelemente aus Aufgabe 1
        \item Zeitkomplexität der ganzen Teilaufgabe: $O(n)$
    \end{itemize}
    
\end{document}

%%% Local Variables:
%%% mode: latex
%%% TeX-master: t
%%% End:
