\documentclass[a4paper,11pt]{scrartcl}

\usepackage[ngerman]{babel}
\usepackage{lmodern}
\usepackage{ucs}
\usepackage[utf8]{inputenc}
\usepackage{amsmath}
\usepackage[T1]{fontenc}
\usepackage{graphicx}

\title{Laufzeitkomplexitäten}
\author{Matrikelnr. 381165}
\date{23.06.2015}

\begin{document}
\maketitle

Wenn nicht anders angegeben, haben die Matrizen $m$ Zeilen und $n$ Spalten, 
bei quadratischen Matrizen jeweils $n$ Zeilen und Spalten.

\section{Transposition}
Die \texttt{trp}-Methode setzt sich aus fünf Teilen zusammen:

\begin{itemize}
    \item Kopieren der eigenen Instanz
    \item Löschen des Inhalts der eigenen Instanz
    \item Neuanlegen von Zeilen und Spalten in der eigenen Instanz
    \item umgekehrtes Versehen mit Werten der Elemente der eigenen Instanz
    \item Löschen der Kopie der eigenen Instanz
\end{itemize}

Da alle Teile der Methode jeweils sämtliche Matrixelemente durchlaufen, hat 
diese eine Komplexität von $O(mn)$, im Falle von quadratischen Matrizen 
dementsprechend $O(n^2)$.

\section{Addition}
Es wird jedes einzelne Matrixelement um den Wert eines zu addierenden 
Elements erhöht oder verringert. Damit ergibt sich auch hier eine Komplexität 
$O(mn)$ – bei quadratischen Matrizen $O(n^2)$.

\section{Multiplikation}
Hier habe die erste Matrix $l$ Zeilen und $m$ Spalten, die zweite $m$ Zeilen 
und $n$ Spalten. Für die Multiplikation muss zunächst mit $O(ln)$ eine neue 
Matrix mit den „äußeren Dimensionen“\footnote{das Ergebnis hat so viele 
Zeilen wie die erste und so viele Spalten wie die zweite Matrix, bei $(L 
\times M)$ und $(M \times N)$ sind dies die äußeren Angaben} angelegt werden. 

Bei der Matrixmultiplikation an sich werden drei verschachtelte Schleifen 
durchlaufen, wobei jeweils eine $l$, $m$ respektive $n$ Durchläufe hat. 
Dadurch ergibt sich eine Komplexität $O(lmn)$ bzw. bei quadratischen Matrizen 
$O(n^3)$.

Die Ausgabe und das Löschen der Ergebnismatrix haben wie auch deren 
Erstellung eine Komplexität $O(ln)$.

Insgesamt hat die Multiplikation somit eine Komplexität von $O(lmn)$.

\section{Orthogonalitätstest}
Der Orthogonalitätstest setzt sich zusammen aus einem Kopiervorgang, einer 
Transposition sowie Matrixlöschungen mit $O(mn)$, einer Matrixmultiplikation 
mit $O(lmn)$ und einem Test nach einer Einheitsmatrix, bei dem alle Elemente 
betrachtet werden, mit $O(mn)$.

Da die Matrixmultiplikation am aufwändigsten ist, hat die 
\texttt{ort}-Methode eine Komplexität $O(lmn)$ bzw. $O(n^3)$.

\section{Symmetrietest}
Hierzu wird nur bei der oberen Dreiecksmatrix, also nur ungefähr die Hälfte 
der Elemente, betrachtet. Da dies allerdings nur einen konstanten Faktor 
ausmacht, ergibt sich auch hier eine Komplexität $O(mn)$.

\end{document}
 
