\documentclass[a4paper,11pt]{scrartcl}

\usepackage[utf8]{inputenc}
\usepackage{amsmath}
\usepackage[T1]{fontenc}
\usepackage{graphicx}

\date{03.05.2015}
\author{Benedikt Geißler}

\begin{document}

\section*{Funktionen für P und S}

Da sich die Werte für $P$ und $S$ ungefähr proportional entwickeln (siehe 
Graph, beide Achsen logarithmisch: linear ansteigender Graph bedeutet 
ebenfalls konstante Steigung), lassen 
sich deren Funktionen $P(N)$ und $S(N)$ durch einen Punkt ermitteln:

\begin{align*}
  m_P &= \frac{P_1}{N_1} = \frac{82025}{1048576} \approx 0,0782 \\
  P(N) &= 0,0782 \cdot N \\
  m_S &= \frac{S_1}{N_1} = \frac{2228882}{1048576} \approx 2,1256 \\
  S(N) &= 2,1256 \cdot N
\end{align*}

\section*{Laufzeit}
Die Funktion \emph{erathostenes} kann man in drei Blöcke einteilen:

\begin{itemize}
  \item Initialisierung: Zeile 18 bis 23
  \item Sieben nach Erathostenes: Z. 25 bis 35
  \item Zählen der Primzahlen: Z. 37 bis 40
\end{itemize}

Der erste und der letzte Block haben jeweils eine  Komplexität $O(n)$, da ein 
Feld ungefähr der Größe $N$ komplett durchlaufen wird und auf die einzelnen 
Elemente Operationen konstanter Laufzeit ausgeführt werden.

Im mittleren Block hat die äußere Schleife ungefähr $\sqrt{N}$ Durchläufe 
($m$ läuft von 2 bis zu $\lfloor\sqrt{N}\rfloor$), die innere Schleife 
maximal $N - 4$ Durchläufe mit konstanter Laufzeit. Somit ergibt sich hier 
eine Komplexität von $O(n\sqrt{n}) = O(n^{\frac{3}{2}})$, welche auch die 
obere Komplexitätsschranke der gesamten Funktion darstellt.

\includegraphics{ergebnis.pdf}
\end{document}
